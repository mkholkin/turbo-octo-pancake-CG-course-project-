\ssr{ВВЕДЕНИЕ}

Компьютерная графика---совокупность методов и средств преобразования информации в графическую форму и из графической формы с помощью ЭВМ.

Компьютерные техники, создающие анимации преобразования формы искусственного объекта, называются морфингом или метаморфозами.
Методы морфинга сегодня широко используются в компьютерной графике для моделирования преобразование между двумя совершенно разными объектами или для создания новых форм с помощью сочетание других существующих форм.
Они имеют множество применений, начиная от специальных эффектов в киноиндустрии и других видах изобразительного искусства до медицинской визуализации и научных целей.

Цель работы --- разработка программного обеспечения для визуализации процесса морфинга двух фруктов, представленных низкополигональными моделями без отверстий.
Считать, что источник света --- точечный, и его положение совпадает с положением наблюдателя.
Для достижения поставленной цели необходимо решить следующие задачи:
\begin{itemize}
    \item перечислить основные принципы морфинга трехмерных объектов;
    \item описать использованные техники морфинга трехмерных объектов;% и выбрать наиболее подходящий для поставленной задачи;
    \item проанализировать и выбрать алгоритмы решения основных задач компьютерной графики: удаления невидимых линий и поверхностей, учёта освещения;
    \item спроектировать программное обеспечение для визуализации процесса морфинга двух фруктов;
    \item выбрать средства реализации, описать формат входных и выходных данных, реализовать спроектированное программное обеспечение;
    \item исследовать вклад времени выполенения этапов морфинга в общее время процесса. %TODO: возможно переименовать
\end{itemize}
\clearpage

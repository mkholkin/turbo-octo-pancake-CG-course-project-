\chapter{Аналитическая часть}

\section{Явление морфинга}
    Слово <<морфинг>> происходит от слова <<метаморфоза>>, которое, согласно 
    Согласно Оксфордскому словарю~\cite{oxford_dictionary}, имеет следующее значение: "Процесс, в ходе которого кто-то/что-то полностью превращается во что-то другое".

    Таким образом, в случае трехмерных объектов термин <<морфинг>> можно интерпретировать как построение последовательности кадров, соответствующей постепенному переходу между двумя различными
    объектами, так называемыми исходными (начальными) и целевыми (конечными) моделями. На рисунке~\ref{fig:morhping_example} представлен пример морфинга трехмерных объектов.
    
    \begin{figure}[H]
		\centering
    	\includegraphics[width=\textwidth]{../inc/images/morhping_sequence}
    	\caption{Пример последовательности кадров морфинга}
    	\label{fig:morhping_example}  
    \end{figure}

\section{Основные этапы морфинга}
	Поскольку фрукты в рамках данной работы представленны низколигональными трехмерными объектами без отверстий, что топологически эквивалентно сфере TODO: можно вообще такое писать без пояснения??, все по этапы будут рассмотрены для подобных объектов.
	
 
    \subsection{Установление соответствия между объектами}

    Ключевой и наиболее сложной задачей в процессе морфинга является установление соответствия между исходным и целевым объектами. Необходимо определить, как точки, ребра и грани одной модели соотносятся с элементами другой.
    
    \subsection{Параметризация на сферу}
    Чтобы сопоставить одну модель другой, необходимо выполнить их отображение в общее параметрическое пространство, а именно, единичную сферу. TODO: Как не закопаться?????
    
	\subsection{Параметризация звездообразных тел}
	\textit{Звездообразным} называется тело, которое имеет хотябы одну внутреннюю точку такую, что отрезки, соединяющие ее с вершинами тела, полностью лежат внутри фигуры \cite{alexa}.
	   
   Пусть точка $O$ видна из всех вершин тела $A$. Тогда, чтобы параметризовать тело $A$, необходимо перенесити его так, чтобы точка $O$ совпала с началом координат и нормировать координаты всех вершин.

	\subsection{Метод релаксации}
	Для параметризации произвольных объектов, топологически эквивалентных сфере, применяется итерационный метод релаксации \cite{alexa}. Метод основан на итеративном уточнении положений вершин.
	Процесс релаксации представлен на рисунке~\ref{fig:relaxation}. 
	
	В качестве исходного состояния создается грубая проекция модели на единичную сферу. Для этого вычисляется произвольная внутренняя точка объекта, которая принимается за центр сферы, после чего все вершины модели проецируются на ее поверхность. Начальная конфигурация, как правило, содержит значительные искажения и полигоны с некорректной ориентацией.
	
	На каждом раунде релаксации вершины сдвигаются к центру масс своих соседей:
	
	\begin{equation}
		v_i^{k + 1} = \frac{\sum_{j \in N(i)} v_j^{k}}{\parallel \sum_{j \in N(i)} v_j^{k} \parallel}, 
	\end{equation}
	
	где
	\begin{itemize}
		\item $v_i^{k + 1}$ --- новое положение $i$-ой вершины после $k$-го раунда релаксации;
		\item $v_i^{k}$ --- положение $i$-ой вершины на момент $k$-го раунда релаксации;
		\item $N(i)$ --- множество вершин, смежных с $i$-ой.
	\end{itemize}
	
	Для предотвращения коллапса всех вершин в одну точку после каждой итерации выполняется ре-центрирование всей сетки относительно начала координат:
	
	\begin{equation}
		v_i' = v_i - \frac{\sum_{j = 0}^{n} v_j}{n},
	\end{equation}
	
	где
	\begin{itemize}
		\item $v_i'$ --- новое положение $i$-ой вершины;
		\item $n$ --- количество вершин.
	\end{itemize}
	
	Релаксация прекращается, когда все полигоны на сфере приобретают ту же ориентацию, что и на исходной 3D-модели, что гарантирует отсутствие самопересечений и складок на поверхности.
	
	
	\begin{figure}[H]
		\centering
		\includegraphics[width=\textwidth]{../inc/images/relaxation}
		\caption{Процесс релаксации сетки, красными отмечены грани c непрвильной ориентацией}
		\label{fig:relaxation}
	\end{figure}
	
	
    \subsection{Создание общего представления}
    После того как установлено соответствие, необходимо создать единую структуру, которая будет использоваться для всех промежуточных форм. Существует два основных подхода:
    \begin{itemize}
        \item \textit{Создание <<суперсетки>> (Map overlay)}\cite{alexa}: На параметрической сфере сетки накладываются друг на друга. В местах пересечения ребер создаются новые вершины, что формирует объединенную сетку, которая может принимать форму, как исходной так и целевой модели.
        \item \textit{Перестроение сетки (Remeshing)}\cite{alexa}: На основе прпметризации создается новая TODO
    \end{itemize}

    \subsection{Интерполяция геометрии}
    Заключительный этап --- вычисление траекторий движения вершин из их начального положения в конечное. Для каждой вершины общего представления необходимо вычислить ее координаты на исходной и целевой моделях. Процесс генерации промежуточных кадров сводится к интерполяции этих координат.

    Для определения начального и конечного положения каждой вершины общего представления используется метод, основанный на барицентрических координатах. Этот подход позволяет точно перенести положение точки из параметрического пространства на поверхность трехмерной модели.
    Процесс вычисления состоит из следующих шагов:
    \begin{itemize}
        \item поиск содержащего треугольника: Для каждой вершины общего представления определяется треугольник исходной (или целевой) сетки, в который она попадает в общем параметрическом пространстве (на сфере);
        \item расчет барицентрических координат: Вычисляются барицентрические координаты данной вершины относительно вершин найденного треугольника;
        \item проецирование в мировое пространство: полученные барицентрические координаты применяются к вершинам соответствующего треугольника, но уже на реальной модели.
    \end{itemize}

    После того как для всех вершин определены их начальные и конечные положения, для генерации промежуточных состояний морфинга применяется линейная интерполяция. Положение $P(t)$ каждой вершины в момент времени $t$, где t изменяется от 0 до 1, вычисляется по формуле:
    \begin{equation}
        P(t) = (1 - t) \cdot P_0 + t \cdot P_1,
    \end{equation}

    где
    \begin{itemize}
        \item $P_0$ --- начальное положение вершины,
        \item $P_1$ --- конечное положение вершины.
    \end{itemize}
    
    \section{Формализация объектов сцен}
    
    На сцене выбора начальных и конечных объектов могут присутствовать 2 фрукта: начальный и конечный. Фрукт (объект) задатся моделью описания объектов, данными для этой модели, оптическими характеристиками поверхности.
    
    На сцене просмотра морфинга находится результат морфинга (задается исходным и целевым объектами, стадией морфинга). 
    
    На обеих сценах могут присутствовать следующие типы объектов:
    \begin{itemize}
    	\item источник света (задается положением в пространстве, цветом и интенсивностью света);
    	\item наблюдатель (задается положением в пространстве, точкой в пространстве, на которую направлен взгляд, направлением верха обзора TODO);
    \end{itemize}
    
    \section{Выбор модели описания объекта}
    Наиболее распространенные модели описания трехмерных объектов в компьютерной графике: \textit{аналитичкская}, \textit{полигональная модель}, \textit{воксельная модель}~\cite{porev}.
    
    \subsection{Аналитическая модель}
    Аналитическая модель представляет собой описание поверхности математическими формулами~\cite{porev}. Обычно поверхность задается уравнением вида $z=f(x,y)$ или $F(x,y,z)=0$.
    
    Отличительные черты~\cite{porev}:
    \begin{itemize}
    	\item легкая процедура рассчета координат точек, нормалей;
    	\item небольшой объем информации для описания форм;
    	\item сложные формулы, которые медленно вычисляются компьютером;
    	
    	% TODO: Начиная отсюда идет отсебятина, а не источник, это номр???
    	\item задание объекта набором поверхностей, если невозможно описать его аналитически;
    	\item отсутствие погрешности при задании сферического объекта; 
    \end{itemize}
    
    \subsection{Полигональная модель}
    В полигональной модели информация об объекте сосотоит из следущих компонентов~\cite{porev}:
    \begin{itemize}
	\item вершина --- точка ($(x, y, z)$ в декартовой системе координат);
	\item отрезок прямой --- задается двумя вершинами;
	\item полилиня --- задается несколькими отрезками прямой;
	\item полигон --- описывает плоскую грань объемного объекта в виде замкнутой линии;
	
	Несколько граней (полигонов) составляют объемный объект в виде полигональной поверхности, также назваемой <<полигональной сеткой>>.
    \end{itemize}
    
    \subsection{Воксельная модель}
    TODO
    
    \subsection{Сравнение моделей описания объектов}
    TODO: а надо ли мне вообще выбирать, если у меня в ТЗ написанно НИЗКОПОЛИГОНАЛЬНЫЕ ОБЪЕКТЫ????
    TODO: можно сравнить по критериям: сложность вычисления нормали, изображение любых объектов, сохранение качества при увеличении
    
    Среди всех моделей наиболее подходящей является полигональная модель, т.~к. с помощью нее можно описать объекты любой сложности, и сохраняет качестве при увеличении, что необходимо в задаче интерактивной визуализации морфинга,  поэтому использованная будет именно она.
    
    \section{Анализ алгоритмов удаления невидимых линий и поверхностей}
    
    Алгоритмы удаления невидимых линий и поверхностей служат для удаления ребер, поверхностей или объемов, которые видимы или невидимы для наблюдателя, находящегося в заданной точке пространства\cite{rogers}.
    
    Самыми распространенными яляются: \textit{алгоритм Робертса}, \textit{алгоритм, использоующий z-buffer}, \textit{алгоритм трассировки лучей}.
    
   \subsection{Алгоритм Робертса}
   
   Данный алгоритм применим только к выпуклым телам. Если обрабатываемое тало невыпуклое --- его необходимо предварительно разбить на выпуклые~\cite{rogers}.
   
   Алгоритм состоит из следующих этапов~\cite{rogers}:
   \begin{enumerate}
   	\item[1)] Удаление граней, экранируемых самим телом.
   	\item[2)] Удаление граней, экранируемых другими телами.
   	\item[3)] Удаление линий пересечения тел, экранируемых самими телами.
   \end{enumerate}
   
   Ассимптотическая оценка трудоемкости: $O(N^2)$, где $N$ --- количество граней. TODO: проверить
   
   \subsection{Алгоритм, использующий \textit{z-buffer}}
   
   Идея \textit{z}-буфера является обобщением идеи буфера кадра. Буфер кадра используется для запоминания аттрибутов (интенсивности) каждого пикселя в пространстве изображения. \textit{z}-буфер --- это отдельны буфер глубины, используемый для запоминания координаты \textit{z} каждого видимого пикселя в пространстве изображения\cite{rogers}.
   
   Этапы работы алгоритма~\cite{rogers}:
   
   \begin{enumerate}
   	\item[1)] Заполнить буфер кадра фоновым значением.
   	\item[2)] Заполнить z-буфер минимальным значением глубины.
   	\item[3)] Преобраз TODO: переход в куб камеры.
   	\item[4)] Для каждого пикселя ($x$, $y$), пренадлежащего телу вычислить его глубину $z(x, y)$.
   	\item[5)] Если глубина $z(x, y)$ > $z\textit{-буфер}(x, y)$, то записать атрибут текущего тела в $\textit{буфер-кадра}(x, y)$, записать глубину $z(x, y)$ в $z\textit{-буфер}(x, y)$.
   \end{enumerate}
   
   
   Асимптотическая оценка трудоекости: $O(N)$, где $N$ --- количество граней. TODO: проверить
      
   \subsection{Алгоритм художника}
   
   Идея алгоритма состоит в том, чтобы подобно художнику отрисовывать объекты по мере их приближения к наблюдателю.
   
   Основные этапы алгорима~\cite{rogers}:
   \begin{enumerate}
   	\item[1)] Отсортировать грани по минимальному или максимальному значению глубины.
   	\item[2)] Отрисовать грани в отсортированном порядке.
   \end{enumerate}
   
   Простая сортировка не всегда дает корректный список приоритетов, тогда приходится использовать дополнительные методы разрешения конфликтов~\cite{rogers}. 
   
   Алгоитм не справляется со случаями циклического перекрытия и пересечения многоугольников.
   
   Ассиптотическая оценка трудоемкости: $O(N)$, где $N$ --- количество граней, однако стоит дополнительно учитывать трудоемкость предварительной сортировки.
     
    \subsubsection{Алгоритм трассировки лучей}
    
    Ассимптотическая оценка трудоемкости: $O(WHN)$, где $W$ --- ширина экрана в пикселях, $H$ --- высота экрана в пикселях, $N$ --- количество граней.   
    
    \subsection{Сравнение алгоритмов удаления невидимых линий и поверхностей}
    
    В таблице~\ref{tbl:} представленные результаты сравнения алгоритмов и использованы следующией обозначения:
    
    \begin{itemize}
    	\item Р --- алгоритм Робертса;
    	\item ZБ --- алгоритм, использующий \textit{z}-буфер;
    	\item Х --- алгоритм Художника;
    	\item ТЛ --- алгоритм трассировки лучей.
  	\end{itemize}
  	
\begin{table}[h!]
	\centering
	\caption{Сравнительная таблица}
	% Определяем новый тип столбца C для центрирования в ячейке с заданной шириной и по вертикали
	\newcolumntype{C}[1]{>{\centering\arraybackslash}m{#1}}
	% Определяем новый тип столбца J для выравнивания по ширине и по вертикали
	\newcolumntype{J}[1]{>{\justifying\arraybackslash}m{#1}}
	
	% Ширина первого столбца ~35% от ширины текста
	% Ширина остальных 6 столбцов ~10.8% каждая
	\begin{tabular}{|J{0.35\textwidth}|C{0.108\textwidth}|C{0.108\textwidth}|C{0.108\textwidth}|C{0.108\textwidth}|C{0.108\textwidth}|C{0.108\textwidth}|}
		\hline
		& \textbf{Р} & \textbf{ЗБ} & \textbf{Х} & \textbf{В} & \textbf{ВА} & \textbf{ОТ} \\
		\hline
		Возможность построения отражений и преломлений & - & - & - & - & - & + \\
		\hline
		Возможность использования без сортировки & + & + & - & + & - & + \\
		\hline
		Возможность использования для аналитических объектов & - & + & - & - & - & + \\
		\hline
		Временная сложность & $O(N^2)$ & $O(WHN)$ & $O(N)$ & $O(WHN)$ & $O(N^2)$ & $O(WHN)$ \\
		\hline
	\end{tabular}
\end{table}

  	
    
    \section{Модель освещения}
    
    В компьютерной графике наиболее распространненными являются две модели освещения: \textit{локальную} и \textit{глобальную}~\cite{rogers}.
    
    Локальная модель учитывает только свет, падающий от источника (источников), и ориентацию поверхности~\cite{rogers}.
    
    Глобальная модель освещения учитывает также свет, отраженный от других объектов сцены или пропущенный через них~\cite{rogers}.
    
    Поскольку на сцене будет находиться только один объект, то будет использована локальная модель освещения.
    
    Интенсивность $I$ в точке $P$ в локальной модели вычисляется по формуле~\ref{eq:local_light_model}~\cite{rogers}.
    
    \begin{equation}
    	\label{eq:local_light_model}
    	I = k_aI_a + \frac{I_l}{d + K}[k_d(\mathbf{\hat{n}\cdot \hat{L}})+k_s(\mathbf{\hat{R}\cdot \hat{S}})^\alpha],
    \end{equation}
    
    где \\
    $k_a$ --- коэффициент \\
    $I_a$ --- интенсивность фонового освещения; \\
    $I_l$ --- интенсивность источника света; \\
    $d$ --- расстояние от источника света до точки~$P$; \\
    $K$ --- TODO; \\
    $k_d$ --- коэффициент диффузного отражения поверхности; \\
    $\mathbf{n}$ --- вектор нормали к поверхности в точке $P$; \\
    $\mathbf{L}$ --- вектор, обратный вектору падения луча; \\
    $k_s$ --- коэффициент зеркального отражения поверхности; \\
    $\mathbf{R}$ --- вектор, отраженного луча; \\
    $\mathbf{S}$ --- вектор, направленный на наблюдателя из точки $P$; \\
    $\alpha$ --- степень, аппроксимирующая пространственное распределение зеркально отраженного света.
        
    \section{Анализ алгоритмов закраски}
    
    Основными алгоритмами закраски в комьютерной графике являются: \textit{однотонная закраска}, \textit{закраска методом Гуро}, \textit{закраска методом Фонга}~\cite{rogers}. 
    
    \subsection{Однотонная закраска}
    При однотонной закраске для каждой грани (многоугольника) полигональной поверхности вычисляется один уровень интенсивности, с которым закрашивается вся гранть. В результате такой закраски изображенный состоит из отдельных иногоугольников и объект выглядит, как многогранник~\cite{rogers}.
    
    \subsection{Закраска методом Гуро}
    Этот метод предназначен для создания иллюзии гладкой криволинейной поверхности, описанной в виде многогранников или полигональной сетки с плоскиими гранями\cite{rogers,porev,foley}
    
    Метод Гуро основан на интерполяции интенсивности каждого пикселя при закраске. Закрашивание граней по методу гуро осуществляется в четыре этапа~\cite{porev}:
    
    \begin{enumerate}
    	\item [1)] Определение нормали к каждой грани.
    	\item [2)] Определение нормалей в вершинах путем усреднения нормалей прилежащих граней.
    	\item [3)] На основе нормалей в вершинах вычисляются значения интенсивностей в вершинах согласно выбранной модели освещения.
    	\item [4)] Закрашиваются полигоны граней цветом, соответствующим интерполяции значений интенсивности в вершинах.
    \end{enumerate}
    
    \subsection{Закраска методом Фонга}
    
    Аналогичен методу Гуро, но при использовании метода Фонга для определения цвета в каждой точке интерполируются не интенсивности отраженного света, а векторы нормалей. При этом значение интенсивность вычисляется в каждом внутреннем пикселе грани~\cite{rogers,porev}.
    
    \subsection{Сравнение алгоритмов закраски}
    
    Визуальное сравнение алгоитмов представленно на рисунке~\ref{fig:shading_comparison}.
    
    \begin{figure}[H]
    	\centering
    	\includegraphics[width=1\textwidth]{../inc/images/shading_2}
    	\caption{Визуальное сравнение методов закрски}
	    \label{fig:shading_comparison}
    \end{figure}
    
    Поскольку объекты сцены (фрукты) представлены низкополигональными объектами, предпочтительной является однотонная закраски, обеспечивающая дискретность освещения между смежными гранями. В результате мофринга часто возникает множество полигонов, составляющих одну грань, из-за чего при однотонной закраске возникают световые артефакты, как на рисунке~\ref{fig:gouraund_midified},~a.
    
    Для устранения этой проблемы будет использована модифицированная версия закраски Гуро: значение интенсивности вершин в пределах одной грани вычисляется с помощью нормали к этой грани, а не нормалей в вершинах. Это обеспечивает равномерное распределение интенсивности по грани и предотвращает сглаживание рёбер, что продемонстрированно на рисунке~\ref{fig:gouraund_midified},~б.
    
    \begin{figure}[H]
	    \includegraphics[width=\linewidth]{../inc/images/gouraund_modified}
	    \caption{Результат морфинга с использованием различных алгоритмов закраски: а) однотонная закраска; б) закраска модифицированным методом Гуро}
 	    \label{fig:gouraund_midified}
    \end{figure}
\section*{Вывод}

\clearpage
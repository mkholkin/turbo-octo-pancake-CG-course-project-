\chapter{Исследовательская часть}

\section{Технические характеристики}

Технические характеристики устройства, на котором производились замеры
\begin{itemize}
    \item процессор: Intel i9-9880H 8-core 2.30--4.80 ГГц~\cite{processor};
    \item оперативное запоминающее устройство: 2667 MHz DDR4 2$\times$8 ГБ~\cite{ram};
    \item операционная система: macOS Sequoia 15.7.1~\cite{os}.
\end{itemize}

\section{Порядок проведения исследования}
Максимальное количество раундов релаксации --- $10000$.
Для каждой пары моделей было выполнено не менее 100 независимых измерений процессорного времени каждого этапа, после чего полученные значения были усреднены.
Во время проведения измерений исследуемое устройство было подключено к электросети, из сторонних приложений был запущен только эмулятор терминала \textit{iTerm2}~\cite{iterm2}, а также были отключены \textit{Wi-Fi} и \textit{Bluetooth}.

При измерении времени были использованы следующие модели:
\begin{itemize}
    \item Яблоко --- 78 вершин, 152 полигона;
    \item Груша --- 199 вершин, 394 полигона;
    \item Банан --- 53 вершин, 102 полигона;
    \item Лимон --- 72 вершины, 140 полигона.
\end{itemize}


\section{Результаты исследования}
В таблице~\ref{tab:measurements} представлены результаты измерений времени построения морфинга и выполнения каждого отдельного этапа для даной реализации.

\begin{table}[H]
    \centering
    \caption{Время выполнения этапов морфинга (в наносекундах)}
    \label{tab:measurements}
    \newcommand{\rot}[1]{\rotatebox{90}{#1}}
    \begin{tabular}{|l|r|r|r|r|r|}
        \hline
        \multicolumn{1}{|c|}{\textbf{Пара моделей}} &
        \multicolumn{1}{c|}{\textbf{\rot{Общее время}}} &
        \multicolumn{1}{c|}{\textbf{\rot{Параметризация }}} &
        \multicolumn{1}{c|}{\textbf{\rot{\makecell{Построение\\суперсетки}}}} &
        \multicolumn{1}{c|}{\textbf{\rot{\makecell{Перенос\\вершин}}}} &
        \multicolumn{1}{c|}{\textbf{\rot{\makecell{Перенос\\нормалей}}}}  \\
        \hline
        Лимон - Груша & 36208501 & 639596  & 18366430  & 4040569  & 13161906  \\
        \hline
        Яблоко - Лимон  & 13532119  & 75514  & 7521913  & 1475417  & 4459275  \\
        \hline
        Лимон - Банан & 9370467 & 81964 & 5339853  & 1077027 & 2871624 \\
        \hline
        Яблоко - Груша & 36569032  & 669114 & 3894843   & 12528669  & 19476405  \\
        \hline
        Яблоко - Банан  & 9228840 & 107190 & 960505  & 2555064  & 5606079 \\
        \hline
        Банан - Груша  & 26814225 & 669574 & 3091515  & 9457313  & 13595823 \\
        \hline
    \end{tabular}
\end{table}

На рисунке~\ref{fig:bars} представлена диаграмма, иллюстрирующая вклад каждого этапа в общее время выполнения морфинга.

\begin{figure}[H]
    \centering
    \includegraphics[width=\textwidth]{bar}
    \caption{Относительный вклад каждого этапа в общее выполнения морфинга}
    \label{fig:bars}
\end{figure}

В таблице~\ref{tab:percentage} представлены минимальная, максимальная и средняя доля времени выполнения каждого отдельного этапа в процентах, на основе результатов измерений в таблице~\ref{tab:measurements}.

\begin{table}[H]
    \centering
    \caption{Процентное соотношению времени выполнения этапов}
    \label{tab:percentage}
    \begin{tabular}{|l|r|r|r|}
        \hline
        \multicolumn{1}{|c|}{\textbf{Этап}} & \multicolumn{1}{c|}{\textbf{Мин. \%}} & \multicolumn{1}{c|}{\textbf{Макс. \%}} & \multicolumn{1}{c|}{\textbf{Сред. \%}} \\
        \hline
        Параметризация                      & 0.56                                  & 2.50                                   & 1.45                                   \\
        \hline
        Построение суперсетки               & 50.70                                 & 60.75                                  & 54.67                                  \\
        \hline
        Перенос вершин                      & 10.41                                 & 11.53                                  & 11.02                                  \\
        \hline
        Перенос нормалей                    & 27.69                                 & 36.35                                  & 32.86                                  \\
        \hline
    \end{tabular}
\end{table}

На рисунке~\ref{fig:pie} представлена круговая диаграмма, иллюстрирующая среднее отношение времени выполнения этапов к общему времени выполнения морфинга.

\begin{figure}[H]
    \centering
    \includegraphics[width=\textwidth]{pie}
    \caption{Средний временной вклад каждого этапа в общее выполнение морфинга}
    \label{fig:pie}
\end{figure}

\section{Вывод}
В рамках представленной реализации наиболее затратными по времени этапами являются построение суперсетки (в среднем 55\% времени) и перенос нормалей (около 33\%).
Вместе они занимают почти 90\% общего времени выполнения.
Наименее затратным является этап параметризации (менее 2\%).
Следует отметить, что в зависимости от сложности формы этап может занимать значительно больше времени, однако в рамках настоящего исследования для выбранных входных моделей условия остановки итерационного процесса релаксации достигались быстро.

\clearpage

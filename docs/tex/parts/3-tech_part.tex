\chapter{Технологическая часть}


\section{Средства реализации}
В качестве языка программирования для реализации алгоритмов выбран язык \textit{Rust}~\cite{rustbook}, так как он удовлетворяет всем требования.

Для реализации графического интерфейса использована библиотека \textit{egui}~\cite{egui}.

В качестве среды разработки была выбрана IDE \textit{RustRover}~\cite{rustrover}.


\section{Формат входных и выходных данных}
Исходный и целевой объекты загружаются из файлов формата \textit{.obj}.
Файлы этого формата представляют собой текстовые файлы, в которых каждая строка описывает элемент геометрии.
На листинге~\ref{lst:obj-example} приведен пример содержимого \textit{.obj} файла.

\begin{lstlisting}[language={}, caption={Пример .obj файла}, label={lst:obj-example}]
v 1.0 1.0 0.0
v -1.0 1.0 0.0
v -1.0 -1.0 0.0
v 1.0 -1.0 0.0

vn 0.0 0.0 1.0

f 1//1 2//1 3//1 4//1
\end{lstlisting}

Структура файла состоит из директив, каждая из которых начинается с ключевого слова.
В рамках данной работы используются следующие:
\begin{itemize}
    \item \texttt{v} --- определяет геометрическую вершину и ее координаты (x, y, z).
    \item \texttt{vn} --- определяет нормаль к вершине и ее компоненты (x, y, z).
    \item \texttt{f} --- определяет грань (полигон). После ключевого слова перечисляются индексы вершин и нормалей, составляющих грань, в формате \texttt{v//vn}. Индексация начинается с 1.
\end{itemize}

Выходными данными является \textit{RGB} изображение.


\section{Реализация алгоритмов}
В данном разделе приведены листинги кода, реализующего основные алгоритмы.

\subsection{Реализация алгоритма с использованием \textit{z}-буфера и модифицированной закраски Гуро}
На листингах~\ref{lst:z-buf-1-appendix},~\ref{lst:z-buf-2-appendix} приведена реализация алгоритма с использованием \textit{z}-буфера и модифицированной закраски Гуро.

%\lstinputlisting[
%    caption=Реализация алгоритма с использованием \textit{z}-буфера и модифицированной закраски Гуро (часть 1),
%    label=lst:z-buf-1,
%    firstline=142,
%    lastline=179
%]{render/z_buffer.rs}

%\lstinputlisting[
%    caption=Реализация алгоритма с использованием \textit{z}-буфера и модифицированной закраски Гуро (часть 2),
%    label=lst:z-buf-2,
%    firstline=86,
%    lastline=140
%]{render/z_buffer.rs}

\subsection{Реализация алгоритмов морфинга}

На листинге~\ref{lst:morph-create-appendix} представлена реализация алгоритма построения объекта морфинга из исходной и целевой сеток.
%\lstinputlisting[
%    caption=Реализация алгоритма построения мофринга из исходной и целевой сеток,
%    label=lst:morph-create,
%    firstline=32,
%    lastline=116
%]{objects/morph.rs}

На листингах~\ref{lst:mesh-parametrization}--\ref{lst:relocate-normals} и~\ref{lst:mesh-relaxation-appendix}--\ref{lst:supermesh-2-appendix} представлены реализации алгоритмов основных этапов построения морфинга.

\lstinputlisting[
    caption=Реализация алгоритма сферической параметризации сетки,
    label=lst:mesh-parametrization,
    firstline=143,
    lastline=165
]{utils/morphing.rs}

%\lstinputlisting[
%    caption=Реализация алгоритма релаксации сетки на единичной сфере,
%    label=lst:mesh-relaxation,
%    firstline=74,
%    lastline=128
%]{utils/morphing.rs}

\lstinputlisting[
    caption=Реализация алгоритма построения суперсетки (часть 1),
    label=lst:supermesh-1,
    firstline=549,
    lastline=561
]{utils/morphing.rs}

%\lstinputlisting[
%    caption=Реализация алгоритма построения суперсетки (часть 2),
%    label=lst:supermesh-2,
%    firstline=353,
%    lastline=464
%]{utils/morphing.rs}

\lstinputlisting[
    caption=Реализация алгоритма вычисления положения вершин суперсетки на поверхности объекта по их параметрическим координатам,
    label=lst:relocate-vertices,
    firstline=614,
    lastline=633
]{utils/morphing.rs}

\lstinputlisting[
    caption=Реализация алгоритма переноса нормалей граней суперсетки на объект на основе их соответствия в параметрическом пространстве,
    label=lst:relocate-normals,
    firstline=635,
    lastline=656
]{utils/morphing.rs}


\section{Интерфейс программы}
На рисунках~\ref{fig:gui-1},~\ref{fig:gui-2} представлен интерфейс программы.

\begin{figure}[H]
    \centering
    \includegraphics[width=0.8\textwidth]{gui-1}
    \caption{Пример графического интерфейса программы --- управление исходным объеком}
    \label{fig:gui-1}
\end{figure}

\begin{figure}[H]
    \centering
    \includegraphics[width=0.8\textwidth]{gui-2}
    \caption{Пример графического интерфейса программы --- управления морфингом}
    \label{fig:gui-2}
\end{figure}

Для загрузки объектов используются кнопки <<Загрузить исходный объект>> и <<Загрузить целевой объект>>.

После того как объекты загружены, становятся доступна кнопка <<Создать морфинг>>, по нажатию на которую создастся объекта морфинга.
После того как он создан, на верхней панели станет доступна кнопка <<Морфинг>>.

Кнопки на верхней панели позволяют переключаться между просмотром исходного, целевого объекта и результата морфинга.
Пример приведен на рисунке~\ref{fig:gui-1}.

На вкладке <<Морфинг>> можно управлять стадией морфинга, путем изменения параметра <<Стадия>> и просматривать результат.

Поворот и масштабирование объектов выполняется с помощью мыши: зажатие левой кнопки мыши и движение мыши выполняет поворот объекта, а прокрутка колесика мыши выполняет масштабирование объекта.
Либо при помощи соответствующих кнопок на боковой панели.

Изменение оптических свойств поверхности исходного или целевого объекта осуществляется с соответствующих ползунков на боковой панели.

\section{Функциональное тестирование}
Функциональное тестирование осуществлялось на основе визуального сравнения полученных изображений с ожидаемыми результатами.
В таблице~\ref{tab:tests} приведены результаты функционального тестирования.

\begin{table}[H]
    \centering
    \caption{Результаты функционального тестирования}
    \label{tab:tests}
    \begin{tabularx}{\textwidth}{|c|>{\raggedright\arraybackslash}X|>{\raggedright\arraybackslash}X|>{\raggedright\arraybackslash}X|}
        \hline
        \textbf{№} & \textbf{Входные данные} & \textbf{Ожидаемый результат} & \textbf{Фактический результат} \\
        \hline
        1 & Файл с моделью яблока, файл с моделью груши & Плавная анимация превращения яблока в грушу при увеличении параметра $t$ & Плавная анимация превращения яблока в грушу при увеличении параметра $t$ \\
        \hline
        2 & Файл с моделью лимона, файл с моделью яблока & Плавная анимация превращения лимона в яблоко при увеличении параметра $t$ & Плавная анимация превращения лимона в яблоко при увеличении параметра $t$ \\
        \hline
        3 & Файл с моделью банана, файл с моделью лимона & Плавная анимация превращения банана в лимон при увеличении параметра $t$ & Плавная анимация превращения банана в лимон при увеличении параметра $t$ \\
        \hline
        4 & Файл с моделью груши, файл с моделью банана & Плавная анимация превращения груши в банан при увеличении параметра $t$ & Плавная анимация превращения груши в банан при увеличении параметра $t$ \\
        \hline
    \end{tabularx}
\end{table}

На рисунках~\ref{fig:test-1}--\ref{fig:test-4} приведены последовательности кадров при параметре стадии морфинга $t$ равном $0, 0.25, 0.5, 0.75, 1$, полученных в ходе функционального тестирования.

\begin{figure}[H]
    \centering
    \includegraphics[width=\textwidth]{apple-pear}
    \caption{Результаты теста 1 --- кадры, полученные при параметре стадии морфинга $t$ равном $0, 0.25, 0.5, 0.75, 1$}
    \label{fig:test-1}
\end{figure}

\begin{figure}[H]
    \centering
    \includegraphics[width=\textwidth]{lemon-apple}
    \caption{Результаты теста 2 --- кадры, полученные при параметре стадии морфинга $t$ равном $0, 0.25, 0.5, 0.75, 1$}
    \label{fig:test-2}
\end{figure}

\begin{figure}[H]
    \centering
    \includegraphics[width=\textwidth]{banana-lemon}
    \caption{Результаты теста 3 --- кадры, полученные при параметре стадии морфинга $t$ равном $0, 0.25, 0.5, 0.75, 1$}
    \label{fig:test-3}
\end{figure}

\begin{figure}[H]
    \centering
    \includegraphics[width=\textwidth]{pear-banana}
    \caption{Результаты теста 4 --- кадры, полученные при параметре стадии морфинга $t$ равном $0, 0.25, 0.5, 0.75, 1$}
    \label{fig:test-4}
\end{figure}

Все тесты прошли успешно.

\section*{Вывод}
В этом разделе были описаны средства реализации, описан формат входных и выходных данных, представлены реализации основных алгоритмов, продемонстрирован интерфейс программы.
\clearpage